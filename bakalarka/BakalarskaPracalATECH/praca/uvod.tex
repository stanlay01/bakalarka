\mychapter{0}{Úvod}
Zámerom bakalárskej práce je navrhnúť a implementovať server manažmentu rolí vhodný pre komplexný \emph{Workflow manažment systém}, ktorý by fungoval na princípe Petriho sietí.

Workflow manažment systém ponúka možnosť automatizovať procesy, ktoré sa riadia určitou logickou postupnosťou. Vo firemnej oblasti dokáže zrýchliť a zefektívniť vykonávanie určitej logistiky. WfMS oddeľuje procesnú časť od aplikačnej a dátovej časti, vďaka čomu sa zjednodušuje návrh, úprava a implementovanie procesov.

Dôležitou otázkou pri vytváraní WfMS je správa zdrojov.
Treba určiť, kto môže a naopak, kto nesmie pristupovať ku konkrétnym prostriedkom,
aby sa zachovala dôvernosť informácii a nemohol byť narušený systém. Vo WFSM je
najpoužívanejší spôsob správa zdrojov založená na základe systému rolí.
 
Bakalárska práca pozostáva z troch hlavných kapitol. V prvej kapitole rozoberáme problematiku workflow manažmentu, Petriho sietí a riadenia prístupu. Sústreďujeme sa najmä na riadenie prístupu na základe rolí RBAC - Role Based Access Control, ktoré je najviac využívané vo Workflow systémoch, a to najmä vďaka tomu, že je založený na
organizačnej štruktúre podnikov a poskytuje efektívne riadenie a údržbu právomocí. 

V druhej kapitole prezentujeme návrh aplikácie Workflow  manažment systému a jej architektúru. Systém pozostáva z viacerých častí, ktoré dokopy utvárajú komplexný celok. Každú časť podrobne vysvetľujeme s ohľadom na vzájomnú integráciu.

V tretej kapitole sa zaoberáme  implementáciou serveru manažmentu rolí a používateľov. V tejto časti WfMS rozoberáme serverové riešenie riadenia prístupu na základe rolí. Aplikácia poskytuje dve hlavné funkcionality - mapovanie používateľov na roly vo firme a komunikačné rozhranie pre ukladanie rolí a referencií do databázy. Na konci tejto kapitoly testujeme efektívnosť a funkcionalitu aplikácie.

Na základe praktickej časti a jej testovania sme prišli na viaceré možné vylepšenia, ktoré bude možné aplikovať v budúcom vývoji Workflow manažment systému. 




 




















