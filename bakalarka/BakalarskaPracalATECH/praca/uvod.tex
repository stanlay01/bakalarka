\mychapter{0}{Úvod}
Systém vývoja aplikačných programov sa veľmi rýchlo mení. So stúpajúcou zložitosťou počítačových systémov a zvyšovaním počtu užívateľov sa vynára potreba rozložiť aplikácie do viacerých navzájom nezávislých modulov. Najprv sa vytvorili databázy , ktoré umožnili oddeliť dáta od aplikácie. Následne sa v 80-ych rokoch analogickým spôsobom  oddelilo užívateľské rozhranie od samotnej aplikácie. Vznikla tak MVC (model-view-controller) architektúra. Ďalším stupňom vývoja je potreba oddeliť všeobecnú funkcionalitu od aplikácie. Odčlení sa tak procesná, aplikačná a dátová časť. Túto myšlienku poskytuje workflow management systém (WFMS).

Workflow management systém umožňuje  ľahko oddeliť procesy od vizuálnej a dátovej časti, vďaka čomu je upravovanie a znovupoužitie daných procesov jednoduchšie. Je možné si teda predstaviť, že dve rôzne firmy v rovnakej oblasti by využívali rovnaké procesy, pričom by mali osobitnú databázovú aj vizuálnu stránku.

Dôležitou otázkou pri tvorbe workflow management systémou je správa zdrojov. Treba určiť kto môže a naopak kto nesmie pristupovať ku konkrétnym prostriedkom, aby sa zachovala dôvernosť informácii a nemohol byť narušený systém.
Vo WFSM je najpoužívanejší spôsob správa zdrojov na základe systému rolí Role-based access control (RBAC). Tento model sa využíva kvôli tomu, že je založený na organizačnej štruktúre podnikov a poskytuje efektívne riadenie a údržbu právomocí. Vo WFSM je potreba oddeliť  tvorbu procesov od jej používania. RBAC dokáže zabezpečiť previazanie medzi procesmi a konkrétnymi používateľmi. Okrem toho má RBAC ďaľšie výhody ako napríklad uľahčenie administrácie priradením užívateľov k jednotlivým roliam v porovnaní so správou práv pre každého jednotlivého užívateľa osobitne. 

Jedna z možností ako modelovať WFMS je prostredníctvom Petriho sietí. Petriho sieť je matematický nástroj na modelovanie a simulovanie diskrétnych procesov.  Hlavné výhody Petriho sietí sú:
\begin{itemize}
	\item formálna sémantika – proces špecifikovaný Petriho sieťou má precíznu definíciu
	\item grafické zobrazenie – Petriho sieť je grafický jazyk. Dôsledkom tohto Petriho siete
	sú intuitívne a ľahko pochopiteľné, preto sú vhodné aj pri komunikácii
	s koncovými užívateľmi.
	\item expresivita – Petriho siet podporuje všetky primitíva potrebné pre modelovanie
	workflow procesov. Keďže stavy sú reprezentované explicitne, umožňujú
	modelovanie závislostí a implicitné voľby.
	\item analýza – Umožňujú overiť vlastnosti (bezpečnosť, invariantnosť, deadlocky, atď.)
	a vyčísliť výkonové merania (čas odozvy, čas čakania, podiel obsadenosti, atď.)
\end{itemize}



Cieľom tejto práce je na základe štúdia a analýzy workflow managemantu  a Petriho sietí vytvoriť modul na správu rolí a užívateľov pre webovú aplikáciu, ktorá bude poskytovať WFMS na základe Petriho sietí. 

Bakalárska práca pozostáva z 2 hlavných kapitol. Prvá kapitola je zameraná na analýzu workflow systému, Petriho siete a riadenie prístupu vo WFMS. Kapitola sa bližšie zameriava na možné prístupy 

Druhá kapitola je zameraná na opis riešenia celej aplikácie, pričom bližšie rozoberá návrh a implementáciu modulu na priraďovanie rolí k užívateľom a jej nadväznosti k ostatným častiam aplikácie. 


 




















