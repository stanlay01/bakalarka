\mychapter{2}{Záver}
Zámerom práce bolo navrhnúť a implementovať server manažmentu rolí vhodný pre komplexný \emph{Workflow manažment systém}, ktorý by fungoval na princípe Petriho sietí.

\noindent Na začiatku sme analyzovali problematiku Workflow manažmentu, Petriho sietí a používané metódy riadenia prístupu. Rozhodli sme sa následne aplikovať naše vedomosti v aplikácii a využiť pre naše potreby metódu riadenia rolí RBAC, ktorú sme prispôsobili pre náš systém a navrhli spôsob jej implementácie. Následne sme nad danou architektúrou vytvorili aplikáciu pre správu rolí vo firmách a ich  priraďovanie jednotlivým používateľom. Aplikácia je vo forme webovej služby a takisto poskytuje aplikačné rozhranie pre \emph{editor manažmentu rolí a organizačnej štruktúry} pre ukladanie do databázy.

\noindent Vytvorená aplikácia umožňuje efektívne spravovať role vo firme - pridávať a odoberať roly z firmy, mapovať používateľov na roly jednotlivo ale aj skupinovo. Pre tieto účely sa zobrazuje zoznam používateľov, ktorých je možné zotriediť podľa rôznych parametrov, alebo osobitne vyhľadávať na základe textového vstupu. Aplikácia je kompatibilná s ostatnými časťami Workflow systému - používa rovnakú databázu. 

\noindent V budúcnosti bude potrebné vylepšiť model referencii, pridať možnosť vytvárať vo firme hierarchie rolí, prípadne obmedzenia ako statické a dynamické vylúčenie práv  a zabezpečiť bezpečnosť aplikácie tak, aby používateľ upravením klientskej časti skriptu nebol schopný vykonať neoprávnené operácie. 


